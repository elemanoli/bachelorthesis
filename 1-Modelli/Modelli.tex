\chapter{Modelli di campionamento e analisi}







%Ricordiamo che, in questo contesto, i dati che abbiamo a disposizione per prevedere la biodiversità degli ecosistemi sono dati di abbondanza: per una certa frazione di area vengono registrate le specie osservate e il numero di individui per specie presenti nel campione ottenendo così un vettore di abbondanze.
%Dunque vengono raccolte solo le frequenze con cui si presentano le varie specie e queste ultime sono sufficienti per stimare la ricchezza delle specie. 
%In questo schema di campionamento la misura del campione \emph{n} (il numero di individui osservati nell'esperimento) è una variabile casuale e può assumere qualsiasi valore intero.\\


La ricchezza delle specie, cioè il numero di specie, è la misura più intuitiva e più frequentemente usata per caratterizzare la diversità di un dato insieme: questa possiede caratteristiche e proprietà matematiche intuitive ben visibili utili per costruire modelli di comunità. Negli studi biogeografici, le mappe di monitoraggio di specie, flora e fauna locali e regionali forniscono informazioni solo sull'assenza o presenza di una determinata specie in ogni località. Dunque, per questo tipo di studi, la ricchezza delle specie è l'unico dato disponibile per quantificare la diversità di un sistema ed è importante quindi sviluppare dei modelli per ottenere questo tipo di informazione partendo da dei campioni di popolazione ridotti.\\
La ricchezza di specie dipende fortemente dal metodo di campionamento e dalla completezza del campione, il modo di raccogliere le informazioni porta ad avere principalmente due tipi di dati: dati di abbondanza e dati di incidenza.\\
Per fissare la notazione: consideriamo una comunità costituita da N individui appartenenti a S specie distinte. Sia $N_\emph{i}$ il numero di individui della \emph{i}-esima specie con \emph{i}=1,2,...,S, $N_\emph{i}>0$ e $N=\sum_{n=1}^S N_\emph{i}$.
L'abbondaza relativa della specie \emph{i}-esima è $p_\emph{i}=N_\emph{i}/N$, dunque $\sum_{n=1}^S p_\emph{i}=1$. Qui N, S, $N_\emph{i}$ e $p_\emph{i}$ rappresentano i valori veri, ma sconosciuti, dei parametri fondamentali dell'insieme in esame.\\
In base al metodo di campionamento si distinguono due tipi di strutture dati: dati di abbondanza e dati di incidenza\cite{doi:ChaoChiu2016}.

\subsubsection{Dati di abbondanza}
In molti studi biologici o ecologici solitamente gli individui vengono osservati o osservati in un dato momento e vengono classificati in base alla specie di appartenenza. Si prenda ad esempio un campione di \emph{n} individui dall'insieme in esame e si ipotizzi di osservare un totale di $S_\emph{obs}$ specie: questo è il \emph{campione di riferimento}. Questo tipo di data-set può essere ottenuto usando due schemi di campionamento differenti:
\begin{enumerate}
    \item \emph{campionamento di tipo discreto} in cui l'unità campionaria è un individuo. Ad esempio, si campiona un numero fissato \emph{n} di individui in una certa area. Qui la grandezza del campione \emph{n} è fissata e ogni specie può essere rappresentata al massimo da \emph{n} individui;
    
    \item \emph{campionamento di tipo continuo} nel quale il campione viene quantificato misurando su scale continue come tempo, area o volume d'acqua.
    Si fissa, per esempio, una certa area da studiare o un certo periodo di tempo per il quale analizzare il sito in esame. Con questo protocollo di campionamento il numero di individui osservati è una variabile casuale e ogni specie può essere rappresentata da un numero qualsiasi di individui.
\end{enumerate}

\subsubsection{Dati di incidenza}
In alcune indagini le unità di campionamento non sono gli individui, ma porzioni di area o periodi di tempo: queste vengono campionate casualmente e indipendentemente. Ad esempio un'area di interesse può essere divisa in in un certo numero di parti approssimativamente della stessa area, tra queste ne vengono selezionati alcune, in modo casuale, sulle quali eseguire l'indagine.\\
A volte risulta impossibile contare esattamente il numero di individui per ogni specie che appaiono in ogni campione (ad esempio per microrganismi, invertebrati o piante) e quindi viene registrata solo la loro incidenza (presenza o assenza) nel campione. Dunque le stime si basano su degli insiemi di unità di campionamento in cui è registrata solo la presenza o assenza di una certa specie in un dato campione invece che la sua abbondanza.
\\ \\
Avendo a disposizione questo tipo di dati si possono seguire due approcci per stimare la diversità del campione: quello parametrico e quello non parametrico.
In questo lavoro useremo dati di abbondanza ottenuti con campionamento continuo, infatti consideriamo una frazione \emph{a} di un'area A nella quale sono stati registrati il numero di individui presenti in corrispondenza della loro specie di appartenenza. D'ora in poi ci occuperemo solo di questo caso particolare.

\subsubsection{Modelli parametrici e non parametrici}
Negli approcci parametrici che analizzeremo si assume che la distribuzione dell'abbondanza delle specie abbia una certa forma,governata da dei parametri. Facendo il fit della curva dell'abbondanza relativa delle specie dei dati osservati si ottengono i valori dei parametri che,secondo le caratteristiche e le proprietà della distribuzione ipotizzata, permettono di dedurre le informazioni sulla diversità del sistema osservato.\newline
Negli approcci non parametrici, invece, non si fanno assunzioni sulla distribuzione sottostante alla curva dell'abbondanza delle specie. L'intuizione e concetto base su cui si fondano i metodi di stima non parametrici è che le specie abbondanti, cioè quelle a cui appartengono un elevato numero di individui, non danno alcuna informazione sulla ricchezza delle specie inosservate, mentre le specie rare, contengono quasi tutte le informazioni sulla biodiversità. Dunque, la maggior parte degli estimatori non parametrici si basa sulle frequenze di apparizione di basso ordine, specialmente sul numero di \emph{singletons} e \emph{doubletons}, cioè sul numero specie che vengono registrare contenere uno o due individui.

