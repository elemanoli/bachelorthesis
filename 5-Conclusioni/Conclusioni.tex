\phantomsection
%\addcontentsline{toc}{chapter}{Considerazioni finali}
\chapter{Conclusioni}


%------------------------------------------------------------------------------------
%Le conclusioni devono essere brevi e comporsi dei seguenti punti:
%\begin{itemize}
%\item indicazione di ci� che si � esposto e del suo significato
%\item analisi comparativa e commento critico dei risultati presentati
%\item spiegazione motivata delle parti omesse o non approfondite
%\item indicazione dei possibili ulteriori sviluppi.
%\end{itemize}
%------------------------------------------------------------------------------------


% ---------------------  ESEMPI UTILI PRONTI ALL'USO  ----------------------------
%TERZO capitolo della tesi. Esempio di citazione doppia \cite{Munoz-Lipo,Vas}.
%
%Esempio di figura in \figurename\ \ref{FIG:LogoUniPD}.
%
%\begin{figure}[!htbp]
%\centering
%\includegraphics[width=0.25\textwidth]{./figure//LogoUniPD}
%\caption{Esempio di figura.}
%\label{FIG:LogoUniPD}
%\end{figure}
%
%Esempio di tabella in \tablename\ \ref{TAB:Esempio}.
%
%\begin{table}[!htbp]
%\centering
%\renewcommand{\arraystretch}{1.3}
%\caption{Esempio di tabella.}
%\begin{tabular}{cc}
%\hline
%Nome & Valore \\
%\hline
%a & 1 \\
%b & 2 \\
%c & 3 \\
%d & 4 \\
%e & 5 \\
%f & 6 \\
%\hline
%\end{tabular}
%\label{TAB:Esempio}
%\end{table}