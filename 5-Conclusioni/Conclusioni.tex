\phantomsection
%\addcontentsline{toc}{chapter}{Considerazioni finali}
\chapter{Conclusioni}


%------------------------------------------------------------------------------------
%Le conclusioni devono essere brevi e comporsi dei seguenti punti:
%\begin{itemize}
%\item indicazione di ci� che si � esposto e del suo significato
%\item analisi comparativa e commento critico dei risultati presentati
%\item spiegazione motivata delle parti omesse o non approfondite
%\item indicazione dei possibili ulteriori sviluppi.
%\end{itemize}
%------------------------------------------------------------------------------------


% ---------------------  ESEMPI UTILI PRONTI ALL'USO  ----------------------------
%TERZO capitolo della tesi. Esempio di citazione doppia \cite{Munoz-Lipo,Vas}.
%
%Esempio di figura in \figurename\ \ref{FIG:LogoUniPD}.
%
%\begin{figure}[!htbp]
%\centering
%\includegraphics[width=0.25\textwidth]{./figure//LogoUniPD}
%\caption{Esempio di figura.}
%\label{FIG:LogoUniPD}
%\end{figure}
%
%Esempio di tabella in \tablename\ \ref{TAB:Esempio}.
%
%\begin{table}[!htbp]
%\centering
%\renewcommand{\arraystretch}{1.3}
%\caption{Esempio di tabella.}
%\begin{tabular}{cc}
%\hline
%Nome & Valore \\
%\hline
%a & 1 \\
%b & 2 \\
%c & 3 \\
%d & 4 \\
%e & 5 \\
%f & 6 \\
%\hline
%\end{tabular}
%\label{TAB:Esempio}
%\end{table}

In questo lavoro abbiamo analizzato alcuni dei modelli, detti metodi di \emph{upscalig}, nati in ambito ecologico, che permettono di fare previsioni sulla biodiversità di un intero ecosistema pur avendo informazioni solamente della composizione  di una minima parte di questo. Conoscendo il numero di individui, $N^*$, e il numero di specie, $S^*$, presenti alla scala $p^*$ in esame, abbiamo visto che, a seconda del metodo a cui si fa riferimento, può essere stimato il numero di specie, $S$, alla scala globale con:

\begin{itemize}
    \item la (\ref{eq:upscaleNB}) se si assume che la RSA segua una distribuzione binomiale negativa
    
    $$S=S^*\frac{1-(1-\xi)^r}{1-(1-\hat \xi_{\emph{p}}^*)^r}; $$
    
    \item la (\ref{eq:SalphaLS}) o con la  (\ref{eq:upscalingLS}) se si assume che la RSA segua una distribuzione logaritmica di Fisher
    
    $$ S=\hat \alpha \log \left (1+ \frac{N}{\hat \alpha} \right ),$$ 
    
    $$   S=S^*\frac{\log(1-\emph{x})}{\log(1-\emph{$\hat x$}_{\emph{p}^*})};$$
    
    \item la (\ref{eq:chaoworestimator}) se utilizziamo il metodo non parametrico di $Chao_{wor}$
    
    $$ S= S^* +\frac{{S^*_1}^2}{\frac{\emph{n}}{\emph{n}-1}2S^*_2+ \frac{\emph{p}^*}{1-\emph{p}^*}S^*_1}.$$
\end{itemize}

In seguito abbiamo messo in evidenza l'analogia che esiste tra le specie assenti nel campione di riferimento in ecologia, e le specie le cui reads non vengono classificate nell'ambito delle comunità microbiche. 
%Infatti con i metodi di classificazione tassonomica si riescono a ricostruire i DNA delle specie presenti all'interno di un campione, ma non tutte le sequenze trovano corrispondenza nei database in cui sono raccolte le informazioni sul DNA delle specie conosciute.
Alla luce di questa analogia abbiamo applicato i metodi descritti a dati metagenomici riguardanti la popolazione batterica dell'intestino di un individuo sano e di un altro affetto dal morbo di Crohn.\\
I test e le analisi condotte su questi campioni hanno rivelato che:
\begin{itemize}
    \item il metodo $Chao_{wor}$ non può essere applicato a dati di questo tipo;
    \item il metodo della distribuzione logaritmica di Fisher fa buone previsioni se la scala del campione di riferimento non si allontana troppo da quella globale (figure \ref{fig:predSpH} e \ref{fig:predSpC}), ma in nessun caso riesce a riprodurre l'andamento della RSA del sistema in esame (figure \ref{fig:plotRSAH} e \ref{fig:plotRSAC});
    \item il metodo della binomiale negativa fa previsioni ottime (figure \ref{fig:predSpH} e \ref{fig:predSpC}) e riproduce l'andamento della RSA del sistema in esame (figure \ref{fig:plotRSAH} e \ref{fig:plotRSAC}).
    
\end{itemize}

Alcune possibili applicazioni di quanto analizzato in questo lavoro potrebbero essere:

\begin{itemize}

    \item utilizzare il metodo della binomiale negativa per ricostruire l'abbondanza relativa delle specie in comunità microbiche. Questo può portare ad un miglioramento della qualità delle cure somministrate ai pazienti, infatti avere informazioni complete sulla comunità che ospita il batterio causa del disturbo può cambiare e migliorare l'approccio clinico adottato;
    
    \item utilizzare le informazioni che si possono ottenere dalle RSA previste per monitorare i cambiamenti nella composizione del microbioma degli individui, ad esempio in casi di infezione e malattia o durante la somministrazione di antibiotici;
    
    \item studiare le differenze nelle popolazioni di comunità batteriche di individui con diverse storie cliniche.
    

\end{itemize}

%Una possibile applicazione di quanto analizzato in questo lavoro potrebbe essere quella di utilizzare i metodi di \emph{upscaling}, già fortemente affermati in ecologia, per ricostruire l'abbondanza relativa delle specie in comunità microbiche. Questo può portare a dei miglioramenti delle cure somministrate ai pazienti, infatti informazioni complete sulla comunità che ospita il batterio causa del disturbo può cambiare e migliorare l'approccio clinico adottato. 

Per concludere, con questo lavoro abbiamo voluto mostrare come i metodi di \emph{upscaling}, sviluppati e molto utilizzati in ambito ecologico, possano dare interessanti contributi anche nell'ambito delle comunità batteriche. 