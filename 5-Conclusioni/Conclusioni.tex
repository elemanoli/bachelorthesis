\phantomsection
%\addcontentsline{toc}{chapter}{Considerazioni finali}
\chapter{Conclusioni}

In questa tesi sono stati presentati i pi� recenti ed efficaci attacchi di tipo DoS che possono essere usati nella rete di anonimato Tor. Sono stati inoltre discussi dei possibili metodi di difesa per ogni tipo di attacco. Ogni tecnica di difesa cerca di risolvere il relativo attacco e contemporaneamente di non modificare in maniera radicale il funzionamento del protocollo. Attacchi come il blanket blocking non causano malfunzionamenti della rete, contrariamente al CellFlood Attack o allo Sniper Attack che possono causare un calo di prestazioni della stessa e conseguentemente una diminuzione del livello di sicurezza. Come � stato discusso, il primo risulta aggirabile abbastanza facilmente con l'uso dei bridge o dei tool che permettono di offuscare il protocollo Tor, mentre gli attacchi con un obiettivo specifico richiedono tecniche di difesa pi� avanzate, come client puzzle e gli altri metodi visti nel capitolo 4. Inoltre il blanket blocking va a colpire una cerchia pi� o meno ristretta di utenti, mentre gli altri attacchi possono avere conseguenze anche su scala globale. Recentemente non sono state scoperte nuove vulnerabilit� nel protocollo che che possono essere sfruttate per condurre attacchi di questo tipo, ma non � da escludere la possibilit� che ne vengano individuate di nuove in futuro.
%------------------------------------------------------------------------------------
%Le conclusioni devono essere brevi e comporsi dei seguenti punti:
%\begin{itemize}
%\item indicazione di ci� che si � esposto e del suo significato
%\item analisi comparativa e commento critico dei risultati presentati
%\item spiegazione motivata delle parti omesse o non approfondite
%\item indicazione dei possibili ulteriori sviluppi.
%\end{itemize}
%------------------------------------------------------------------------------------


% ---------------------  ESEMPI UTILI PRONTI ALL'USO  ----------------------------
%TERZO capitolo della tesi. Esempio di citazione doppia \cite{Munoz-Lipo,Vas}.
%
%Esempio di figura in \figurename\ \ref{FIG:LogoUniPD}.
%
%\begin{figure}[!htbp]
%\centering
%\includegraphics[width=0.25\textwidth]{./figure//LogoUniPD}
%\caption{Esempio di figura.}
%\label{FIG:LogoUniPD}
%\end{figure}
%
%Esempio di tabella in \tablename\ \ref{TAB:Esempio}.
%
%\begin{table}[!htbp]
%\centering
%\renewcommand{\arraystretch}{1.3}
%\caption{Esempio di tabella.}
%\begin{tabular}{cc}
%\hline
%Nome & Valore \\
%\hline
%a & 1 \\
%b & 2 \\
%c & 3 \\
%d & 4 \\
%e & 5 \\
%f & 6 \\
%\hline
%\end{tabular}
%\label{TAB:Esempio}
%\end{table}