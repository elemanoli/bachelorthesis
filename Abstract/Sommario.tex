\chapter{Abstract}
Il tema della segretezza nelle comunicazioni � diventato sempre pi� di maggiore rilievo negli ultimi anni. Intercettazioni, limitazione della libert� di espressione, censura da parte dei governi e in generale tutto ci� che limita lo scambio di informazioni tra entit� hanno portato ad un crescente interesse verso le comunicazioni protette. Per questi motivi sono stati sviluppati numerosi sistemi che permettono comunicazioni in forma anonima e cifrata, tra questi Tor. � importante comprendere che anche se Tor sta diventando sempre pi� popolare, i suoi utenti sono potenziali obiettivi di attacchi che mirano ad identificarli o impedirne l'utilizzo.
In questa tesi verranno discussi i principali attacchi di tipo \emph{Denial of Service} (DoS) che possono essere condotti nella rete di anonimato Tor. Nella prima parte verranno introdotte delle nozioni basilari di crittografia, verr� spiegato cos'� la rete Tor e come funziona. Successivamente verr� fatta una panoramica sugli attacchi e su possibili contromisure per ognuno. 

% -------------------------------- TRACCIA GUIDA PER LA STESURA ------------------------------
%Il sommario � un breve riassunto dell'elaborato, orientativamente di circa 200 parole. In
%esso il laureando deve esporre concisamente:
%
%\begin{itemize}
%\item il problema che � stato considerato
%\item come il problema � stato risolto
%\item i principali risultati e il relativo significato.
%\end{itemize}
%
%Il sommario deve essere informativo e non una semplice lista di argomenti svolti; da una
%sua lettura, con una preparazione media sull'argomento, si dovrebbe capire se il lavoro �
%di interesse per chi si accinge a consultare la tesi.