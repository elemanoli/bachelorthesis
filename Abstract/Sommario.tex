\chapter{Abstract}
 

% -------------------------------- TRACCIA GUIDA PER LA STESURA ------------------------------
%Il sommario � un breve riassunto dell'elaborato, orientativamente di circa 200 parole. In
%esso il laureando deve esporre concisamente:
%
%\begin{itemize}
%\item il problema che � stato considerato
%\item come il problema � stato risolto
%\item i principali risultati e il relativo significato.
%\end{itemize}
%
%Il sommario deve essere informativo e non una semplice lista di argomenti svolti; da una
%sua lettura, con una preparazione media sull'argomento, si dovrebbe capire se il lavoro �
%di interesse per chi si accinge a consultare la tesi.


In ecologia, un problema nella caratterizzazione della biodiversità di un ecosistema è quello di stimarla utilizzando solo campioni locali, i quali coprono solamente una minima percentuale dell'area su cui si estende il sistema in esame. In questa tesi, seguendo il lavoro presentato nell'articolo "Upscaling species richness and abundances in tropical forests"\cite{Tovoe1701438}, recentemente pubblicato su Science Advance, verranno presentati alcuni dei metodi più utilizzati per superare tale problema ponendo l'accento su come questi derivino naturalmente da principi primi alla base dei processi biologici. Un problema analogo si presenta anche nello studio delle comunità microbiche.\newline
Dopo aver testato l'affidabilità dei modelli in questo ambito, verranno applicati a dei dati presi dallo studio "Characterization of the gut microbiome using $16S$ or shotgun metagenomics"\cite{shotgun}.