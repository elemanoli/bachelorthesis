\chapter{Abstract}
 

% -------------------------------- TRACCIA GUIDA PER LA STESURA ------------------------------
%Il sommario � un breve riassunto dell'elaborato, orientativamente di circa 200 parole. In
%esso il laureando deve esporre concisamente:
%
%\begin{itemize}
%\item il problema che � stato considerato
%\item come il problema � stato risolto
%\item i principali risultati e il relativo significato.
%\end{itemize}
%
%Il sommario deve essere informativo e non una semplice lista di argomenti svolti; da una
%sua lettura, con una preparazione media sull'argomento, si dovrebbe capire se il lavoro �
%di interesse per chi si accinge a consultare la tesi.


Uno dei problemi aperti in Ecologia è quello di caratterizzare la biodiversità di un ecosistema stimandola
attraverso censimenti locali, che tipicamente coprono solamente una minima percentuale dell'area su cui si estende il sistema in esame.
In questa tesi, verranno presentati alcuni dei metodi proposti in letteratura per superare tale problema, con particolare attenzione ad un modello di upscaling presentato nell'articolo "Upscaling species richness and abundances in tropical forests", recentemente pubblicato su Science Advances \cite{Tovoe1701438}. \newline
Dopo aver testato l'affidabilità dei modelli proposti, applicheremo per la prima volta il metodo di upscaling per lo studio della biodiversità in comunità microbiche, utilizzando dati di metagenomica relativi ai batteri dell'intestino umano \cite{shotgun}.