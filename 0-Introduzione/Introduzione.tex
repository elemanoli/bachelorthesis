\chapter*{Introduzione}
\addcontentsline{toc}{chapter}{Introduzione}
Tutto ciò che è importante in ecologia è strettamente legato a questioni di scala. Le condizioni fisiche variano in maniera complicata nello spazio e nel tempo e diversi organismi percepiscono, si muovono e reagiscono alle condizioni a diverse scale spaziali. 
%Le diverse popolazioni possono interagire in modi che dipendono dalla scala, 
Le diverse cause dei cambiamenti ecologici (naturali o antropiche) tendono a manifestarsi a scale diverse, e come risultato i cambiamenti nella biodiversità possono essere diversi a seconda della scala. Anche i nostri interessi riguardo ai sistemi naturali variano a seconda della scala; gli obiettivi di conservazione possono riguardare scale globali, nazionali o regionali, mentre questioni di servizi ecosistemici, cioè i benefici forniti dagli ecosistemi al genere umano, riguardano le proprietà delle comunità a scala locale. Infine i nostri tentativi di monitorare, capire e gestire le comunità ecologiche sono strettamente legati e limitati alle scale.\\
Solitamente, per motivi pratici, monitoriamo gli ecosistemi a scale ridotte, ma spesso siamo interessati alle proprietà di una comunità a scale diverse. 
%Ad esempio le misure agro-ambientali  
Non è possibile monitorare un intero paesaggio o un intero continente dunque tipicamente si studiano dei campioni, ma questi danno informazioni solo sulla biodiversità alla scala considerata che non si possono utilizzare direttamente per inferire la biodiversità alla scala totale, in quanto questa non è additiva. Estrapolare la ricchezza delle specie da scala locale a totale non è dunque una cosa semplice e per riuscire in questo intento sono stati sviluppati molti metodi.\\
Uno strumento statistico comunemente usato per descrivere la normalità o rarità della presenza delle specie in una comunità ecologica è la \textbf{relative species abundance distribution} (\textbf{SAD} o \textbf{RSA}) cioè la descrizione delle abbondanze, ovvero il numero di individui, per ogni specie osservata all'interno di una comunità. Precisamente definiamo la SAD come un vettore di abbondanze di tutte le specie presenti in una comunità ecologica. Spesso è rappresentata in un diagramma in cui l'asse \emph{x} rappresenta il logaritmo delle abbondanze e l'asse \emph{y} il numero di specie.


%\emph{anonimato}, \emph{non linkabilit�}, \emph{inosservabilit�}.



%\cite{tor}, nascono per garantire segretezza nelle comunicazioni, in particolare cercando di raggiungere degli obiettivi fondamentali:
%\begin{description}
%\item \textbf{Anonimity}, definita come lo stato di non essere identificabili all'interno di un insieme di soggetti.
%\item \textbf{Unlinkability}.
%\item \textbf{Unobservability}.
%\end{description} 

%i client devono potersi scambiare messaggi in totale anonimato, questo significa che essi non devono essere identificabili all'interno di un insieme di soggetti.


%
%\lipsum[1-4]
%wikileaks

%\subsubsection{ }


           
%\lipsum[1]

%\chapter{Introduzione}
%L'introduzione costituisce la prima parte dell'elaborato ed estende quanto contenuto nel
%sommario, orientando meglio la lettura. In essa vanno inserite le informazioni che stanno a
%monte, logicamente e cronologicamente, al lavoro svolto nella tesi. Si compone
%essenzialmente dei seguenti punti:
%\begin{itemize}
%\item spiegazione della natura del problema considerato
%\item descrizione dei contenuti reperibili in letteratura relativamente al problema in
%questione, corredata da esaurienti citazioni bibliografiche
%\item scopo del lavoro
%\item indicazione dei metodi di soluzione del problema
%\item elenco schematico del contenuto dei vari capitoli.
%\end{itemize}