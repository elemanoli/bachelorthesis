\chapter*{Introduzione}
\addcontentsline{toc}{chapter}{Introduzione}
Uno dei più importanti obiettivi in ecologia è quello di dedurre le proprietà generali di un ecosistema campionando solo una frazione di esso. \\
Per ragioni pratiche la biodiversità viene misurata o monitorata tipicamente a piccole scale, ma è importante poter conoscere la biodiversità dell'intero ecosistema 
\\\
....
\\\\
Uno strumento statistico comunemente usato per descrivere la normalità o rarità della presenza delle specie in una comunità ecologica è la \textbf{relative species abundance distribution} (\textbf{SAD} o \textbf{RSA}) cioè il numero di individui per specie presenti all'interno di una regione. Normalmente la SAD è misurata a scale locali


%\emph{anonimato}, \emph{non linkabilit�}, \emph{inosservabilit�}.



%\cite{tor}, nascono per garantire segretezza nelle comunicazioni, in particolare cercando di raggiungere degli obiettivi fondamentali:
%\begin{description}
%\item \textbf{Anonimity}, definita come lo stato di non essere identificabili all'interno di un insieme di soggetti.
%\item \textbf{Unlinkability}.
%\item \textbf{Unobservability}.
%\end{description} 

%i client devono potersi scambiare messaggi in totale anonimato, questo significa che essi non devono essere identificabili all'interno di un insieme di soggetti.


%
%\lipsum[1-4]
%wikileaks

%\subsubsection{ }


           
%\lipsum[1]

%\chapter{Introduzione}
%L'introduzione costituisce la prima parte dell'elaborato ed estende quanto contenuto nel
%sommario, orientando meglio la lettura. In essa vanno inserite le informazioni che stanno a
%monte, logicamente e cronologicamente, al lavoro svolto nella tesi. Si compone
%essenzialmente dei seguenti punti:
%\begin{itemize}
%\item spiegazione della natura del problema considerato
%\item descrizione dei contenuti reperibili in letteratura relativamente al problema in
%questione, corredata da esaurienti citazioni bibliografiche
%\item scopo del lavoro
%\item indicazione dei metodi di soluzione del problema
%\item elenco schematico del contenuto dei vari capitoli.
%\end{itemize}