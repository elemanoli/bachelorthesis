\chapter*{Introduzione}
\markboth{Introduzione}{Introduzione}
Una sfida cruciale per la ricerca in ecologia è quella di comprendere come le varie grandezze, tra cui la biodiversità e l'abbondanza delle specie, cambiano attraverso le diverse scale spaziali \cite{doi:10.1111/2041-210X.12319}. Infatti, in un ecosistema, le condizioni fisiche variano in maniera complicata nello spazio e nel tempo e le diverse popolazioni possono interagire in modi che dipendono dalla scala.

Le diverse cause dei cambiamenti ecologici, siano esse naturali o antropogeniche, tendono a manifestarsi a scale diverse, e come risultato i cambiamenti nella biodiversità possono essere diversi a seconda della scala a cui li si analizza. Anche i nostri interessi riguardo ai sistemi naturali variano a seconda della scala; gli obiettivi di conservazione possono riguardare scale globali, nazionali o regionali, mentre questioni di servizi ecosistemici, cioè i benefici forniti dagli ecosistemi, riguardano le proprietà delle comunità a scala locale.

Introduciamo ora alcune definizioni che utilizzeremo nel corso della tesi. Con il termine \emph{biodiversità} indicheremo sia la ricchezza di specie, sia la loro abbondanza relativa nello spazio e nel tempo; la \emph{ricchezza di specie} è semplicemente il numero di specie in uno spazio definito ad un dato istante e l'\emph{abbondanza relativa delle specie} si riferisce alla loro rarità o predominanza in termini di individui che compongono la specie stessa relativamente alla popolazione dell'intera comunità ecologica. Definiamo infine una \emph{comunità ecologica} come un gruppo di specie simili a livello trofico che competono o potrebbero competere in un'area per le stesse o per simili risorse. 
 
Solitamente, per motivi pratici, gli ecosistemi vengono monitorati a scale ridotte, ma spesso l'interesse ricade sulle proprietà di una comunità a scale diverse.
%Ad esempio le misure agro-ambientali  
Poiché non è possibile monitorare un intero paesaggio o un intero continente, tipicamente si studiano dei campioni, ma questi danno informazioni solo sulla biodiversità alla scala considerata che non si possono utilizzare direttamente per inferire la biodiversità alla scala globale, non godendo quest'ultima di proprietà additiva. Estrapolare la ricchezza delle specie da scala locale a globale non è dunque una cosa semplice e per riuscire in questo intento sono stati sviluppati molti metodi. Uno strumento statistico comunemente usato per descrivere la predominanza o rarità della presenza delle specie in una comunità ecologica è la \textbf{species abundance distribution} (\textbf{SAD}) cioè la descrizione delle abbondanze, ovvero il numero di individui, per ogni specie osservata all'interno di una comunità\cite{doi:McGill2007}. Precisamente definiamo la SAD come un vettore di abbondanze di tutte le specie presenti in una comunità ecologica. Spesso è rappresentata in un istogramma in cui l'asse \emph{x} rappresenta il logaritmo in base $2$ delle abbondanze e l'asse \emph{y} la frequenza di specie con tale abbondanza\cite{Preston}. Un'altra quantità studiata in ecologia è la \textbf{species area relationship} (\textbf{SAR}), una curva che descrive come cresce il numero di specie (diverse) al crescere dell'area campionata su cui l'ecosistema si estende.


Diamo ora una breve descrizione della \textbf{teoria neutrale}, che è il quadro teorico per la modellazione della dinamica delle popolazioni nel quale si svolge questo lavoro\cite{Hubbell}. Con \emph{neutrale} si intende che la teoria tratta gli organismi di una comunità come identici nella loro probabilità di nascita, morte, riproduzione, migrazione e speciazione. Si utilizza il termine neutrale quindi per descrivere l'ipotesi dell'equivalenza di tutti gli individui appartenenti alle specie di una data comunità ecologica. Notiamo che questa definizione di neutralità non esclude il fatto che gli individui possano interagire anche con processi ecologici complessi. Dunque la caratteristica che definisce una teoria neutrale in ecologia non è la semplicità delle regole che governano le interazioni tra gli individui ma piuttosto la completa identità delle interazioni stesse. Nonostante le sue drastiche ipotesi alla base, si è mostrato che questa teoria descrive molto bene le proprietà emergenti di comunità ecologiche di un dato livello trofico e che sono molto biodiverse \cite{2016AzaeleSuweis}.



Nei prossimi capitoli ci occuperemo di descrivere alcuni tra i metodi che sono stati sviluppati per dedurre la ricchezza delle specie a scala globale a partire da un campione ridotto di SAD di un dato ecosistema. 


%\emph{anonimato}, \emph{non linkabilit�}, \emph{inosservabilit�}.



%\cite{tor}, nascono per garantire segretezza nelle comunicazioni, in particolare cercando di raggiungere degli obiettivi fondamentali:
%\begin{description}
%\item \textbf{Anonimity}, definita come lo stato di non essere identificabili all'interno di un insieme di soggetti.
%\item \textbf{Unlinkability}.
%\item \textbf{Unobservability}.
%\end{description} 

%i client devono potersi scambiare messaggi in totale anonimato, questo significa che essi non devono essere identificabili all'interno di un insieme di soggetti.


%
%\lipsum[1-4]
%wikileaks

%\subsubsection{ }


           
%\lipsum[1]

%\chapter{Introduzione}
%L'introduzione costituisce la prima parte dell'elaborato ed estende quanto contenuto nel
%sommario, orientando meglio la lettura. In essa vanno inserite le informazioni che stanno a
%monte, logicamente e cronologicamente, al lavoro svolto nella tesi. Si compone
%essenzialmente dei seguenti punti:
%\begin{itemize}
%\item spiegazione della natura del problema considerato
%\item descrizione dei contenuti reperibili in letteratura relativamente al problema in
%questione, corredata da esaurienti citazioni bibliografiche
%\item scopo del lavoro
%\item indicazione dei metodi di soluzione del problema
%\item elenco schematico del contenuto dei vari capitoli.
%\end{itemize}